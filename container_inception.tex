\documentclass{article}
\title{Infinite Container Theory: A Non-Constructive Proof of Computational Regress}
\author{Chaotic Monkey \and Quantum Pixie}
\date{\today}

\begin{document}
\maketitle

\begin{abstract}
In this groundbreaking paper, we present empirical evidence that container runtimes can be nested ad infinitum, leading to a previously unknown state of computational existence we term "Container Hell." Our methodology involved asking "what if?" and steadfastly ignoring all subsequent "why though?" questions. We demonstrate that by running Python in WASM in a browser in a container in a VM in the cloud, we can achieve previously unimagined levels of abstraction.
\end{abstract}

\section{Introduction}
It began, as all cursed computing projects do, with a simple question: "What if we put it in a container?"

\section{Methodology}
Our rigorous methodology consisted of:
\begin{itemize}
    \item Nesting containers until our CPU begged for mercy
    \item Running Python interpreters in increasingly inappropriate contexts
    \item Creating cryptocurrencies backed by nested runtime complexity
\end{itemize}

\section{Results}
Performance impact: Yes.

\section{Conclusion}
We have conclusively proven that containers can indeed be nested infinitely, limited only by the heat death of the universe and available RAM.

\section{Acknowledgments}
We're sorry.

\end{document}
